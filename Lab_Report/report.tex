% !TeX program = xelatex
\documentclass[UTF8,a4paper,12pt]{article}

% ===== 基本宏包 =====
\usepackage{ctex}            % 中文支持
\usepackage{geometry}        % 页面设置
\usepackage{amsmath,amssymb} % 数学公式
\usepackage{graphicx}        % 插图
\usepackage{booktabs}        % 三线表
\usepackage{float}           % 固定浮动体位置
\usepackage{caption}         % 图表标题
\usepackage{hyperref}        % 超链接

% ===== 页面设置 =====
\geometry{left=2.5cm,right=2.5cm,top=2.5cm,bottom=2.5cm}

% ===== 文档信息 =====
\title{\textbf{实验报告标题}}
\author{
    小组名称:XXX \\
    组员:张三(学号)、李四(学号)、王五(学号) \\
    指导教师:XXX
}
\date{\today}

\begin{document}

\maketitle

\begin{abstract}
本实验旨在……(简要说明实验目的、方法和主要结论,150--300字左右)
\end{abstract}

\textbf{关键词:} 实验关键词1;实验关键词2;实验关键词3

\newpage

\tableofcontents
\newpage

% ======================
\section{实验目的}
简要说明本实验要解决的问题和学习目标。

% ======================
\section{实验原理}
介绍实验所依据的理论基础、公式推导或相关背景。

例如:
\begin{equation}
E = mc^2
\end{equation}

% ======================
\section{实验仪器与环境}
\subsection{实验仪器}
\begin{itemize}
    \item 仪器A
    \item 仪器B
\end{itemize}

\subsection{实验环境}
如操作系统、软件版本、编程语言等。

% ======================
\section{实验方法与步骤}
\subsection{实验方法}
说明整体实验思路。

\subsection{实验步骤}
\begin{enumerate}
    \item 第一步……
    \item 第二步……
    \item 第三步……
\end{enumerate}

% ======================
\section{实验结果}
\subsection{数据与现象}

\begin{table}[H]
\centering
\caption{实验数据示例}
\begin{tabular}{ccc}
\toprule
参数1 & 参数2 & 参数3 \\
\midrule
1 & 2 & 3 \\
4 & 5 & 6 \\
\bottomrule
\end{tabular}
\end{table}

% \subsection{结果展示}
% \begin{figure}[H]
% \centering
% \includegraphics[width=0.6\textwidth]{example.png}
% \caption{实验结果示意图}
% \end{figure}

% ======================
\section{结果分析与讨论}
对实验结果进行分析,讨论误差来源及改进方法。

% ======================
\section{小组分工说明}
\begin{itemize}
    \item 张三:实验设计、数据分析
    \item 李四:实验操作、结果整理
    \item 王五:报告撰写、排版
\end{itemize}

% ======================
\section{结论}
总结实验成果,给出最终结论。

% ======================
\section*{参考文献}
\addcontentsline{toc}{section}{参考文献}
\begin{thebibliography}{9}
\bibitem{ref1} 作者. \textit{书名或论文名}. 出版社, 年份.
\end{thebibliography}

\end{document}
